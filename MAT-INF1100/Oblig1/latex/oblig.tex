\documentclass{article}
\usepackage[utf8]{inputenc}

\usepackage{mathtools}
\usepackage{amsthm}
\usepackage{cancel}
\usepackage{graphicx}

\usepackage{listings}
\usepackage{xcolor}

\usepackage[T1]{fontenc}

\definecolor{codegreen}{rgb}{0,0.6,0}
\definecolor{codegray}{rgb}{0.5,0.5,0.5}
\definecolor{codepurple}{rgb}{0.58,0,0.82}
\definecolor{backcolour}{rgb}{0.95,0.95,0.92}

\lstdefinestyle{mystyle}{
    backgroundcolor=\color{backcolour},   
    commentstyle=\color{codegreen},
    keywordstyle=\color{blue},
    numberstyle=\tiny\color{codegray},
    stringstyle=\color{codegreen},
    basicstyle=\ttfamily\footnotesize,
    breakatwhitespace=false,         
    breaklines=true,                 
    captionpos=b,                    
    keepspaces=true,                 
    numbers=left,                    
    numbersep=5pt,                  
    showspaces=false,                
    showstringspaces=false,
    showtabs=false,                  
    tabsize=2
}

\lstset{style=mystyle}

\title{Obligatorisk oppgave 1, MAT-INF1100, Høst 2020}
\author{Cory Alexander Balaton}
\date{September 2020}

\begin{document}
\maketitle
\newpage

\section*{Task 1}

\subsection*{a)}
\lstinputlisting[language=Python]{../oppgave_1_a.py}
Under is the printout of the 4 first and 4 last terms of the program:
\begin{verbatim}
    n =   2:                      5.0
    n =   3:                     21.0
    n =   4:                     89.0
    n =   5:                    377.0
                   ...              
    n =  97:      1.8068856563238e+60
    n =  98:     7.65409046775694e+60
    n =  99:    3.242324752735156e+61
    n = 100:   1.3734708057716316e+62
\end{verbatim}

\newpage

\subsection*{b)}
\lstinputlisting[language=Python]{../oppgave_1_b.py}
Under is the printout of the 4 first and 4 last terms of the program:
\begin{verbatim}
    n =   2:      0.05572809000084078
    n =   3:    -0.013155617496426686
    n =   4:    0.0031056200151340363
    n =   5:   -0.0007331374358905407
                   ...              
    n =  97:  -1.5881345077615722e+44
    n =  98:  -6.7274457322911874e+44
    n =  99:  -2.8497917436926322e+45
    n = 100:  -1.2071911547999648e+46
\end{verbatim}

\newpage

\subsection*{c)}
\begin{proof}[\nopunct]
For the equation $x_{n+2} - 4x_{n+1} - x_n = 0$, we can set $x_n = x^n$ and get:
\begin{equation}
    \begin{split}
        x^{n+2} - 4x^{n+1} - x^n &= 0 \\
        x^n(x^{2} - 4x - 1) &= 0 \\
        x &= \frac{4 \pm \sqrt{(-4)^2 - 4 \cdot 1 \cdot (-1)}}{2} \\
        x &= \frac{4 \pm \sqrt{16 + 4}}{2} \\
        x &= \frac{4 \pm \sqrt{20}}{2} \\
        x &= 2 \pm \sqrt{5} \\
        x_1 = 2 + \sqrt{5} & \wedge x_2 = 2 - \sqrt{5}
    \end{split}
\end{equation}

Knowing the roots, we can write the equation:
\begin{equation}
    x_n = C(2 + \sqrt{5})^n + D(2 - \sqrt{5})^n
\end{equation}

We can then find $C$ and $D$ with the initial conditions $x_0 = 1$ and $x_1 = 2 - \sqrt{5}$:
\begin{equation}
    \begin{split}
        C(2 + \sqrt{5})^0 + D(2 - \sqrt{5})^0 &= 1 \\
                                        C + D &= 1 \\
                                            C &= 1 - D
    \end{split}
\end{equation}
\begin{equation}
    \begin{split}
        C(2 + \sqrt{5})^1 + D(2 - \sqrt{5})^1 &= 2 - \sqrt{5} \\
            C(2 + \sqrt{5}) + D(2 - \sqrt{5}) &= 2 - \sqrt{5}
    \end{split}
\end{equation}

By substituting $C$ in equation $(4)$ with $1 - D$, we get:

\begin{equation}
    \begin{split}
                               (1 - D)(2 + \sqrt{5}) + D(2 - \sqrt{5}) &= 2 - \sqrt{5} \\
                        2 - 2D + \sqrt{5} - \sqrt{5}D + 2D - \sqrt{5}D &= 2 - \sqrt{5} \\
          2 - \cancel{2D} + \sqrt{5} + \cancel{2D} - 2 \cdot \sqrt{5}D &= 2 - \sqrt{5} \\
                                                   - 2 \cdot \sqrt{5}D &= - 2 \cdot \sqrt{5} \\
                                                                     D &= 1
    \end{split}
\end{equation}

By substituting $D$ in equation 3 with $1$ we get:

\begin{equation}
    \begin{split}
        C &= 1 - 1 \\
        C &= 0
    \end{split}
\end{equation}

Now that we have found $C$ and $D$, we can substitute them in equation $(2)$ and get:

\begin{equation}
    \begin{split}
        x_n &= 0 \cdot (2 + \sqrt{5})^n + 1 \cdot (2 - \sqrt{5})^n \\
            &= (2 - \sqrt{5})
    \end{split}
\end{equation}
\end{proof}

\subsection*{d)}
When comparing the results of the program in b, and a program that uses the analytical form to calculate $x_n$, the results vary slightly in the beginning because of slight rounding errors.
What was really interesting is when the program in b started to dive downwards in the negative direction at about $n = 13$. If we look at the analytical form, the number should be negative when $n$ is odd, and positive when $n$ is even,
because $x_n = (2-\sqrt{5})^n$ and $2-\sqrt{5} < 0$. Another fact is that $0 < |2-\sqrt{5}| < 1$, which means that $0 < |2-\sqrt{5}|^n < 1$. Since these two statements are not being fulfilled when the program in b is run,
the answers must then be wrong. \\

I wrote an extended program from b where each row shows $x_n$, $x_{n-1}$, and $x_{n-2}$. Program:

\lstinputlisting[language=Python]{../oppgave_1_b_v2.py}

\newpage

And under I have added some of the rows printed that are significant:
\begin{verbatim}
    n =  12:   4*a_11 =   -5.083030174546366e-07 <==> a_10 =    5.374453024842296e-07 
    <==> 4*a_11 + a_10 =    2.914228502959304e-08

    n =  13:   4*a_12 =   1.1656914011837216e-07 <==> a_11 =  -1.2707575436365914e-07 
    <==> 4*a_12 + a_11 =  -1.0506614245286983e-08
    
    n =  14:   4*a_13 =   -4.202645698114793e-08 <==> a_12 =    2.914228502959304e-08 
    <==> 4*a_13 + a_12 =   -1.288417195155489e-08

    n =  15:   4*a_14 =   -5.153668780621956e-08 <==> a_13 =  -1.0506614245286983e-08 
    <==> 4*a_14 + a_13 =   -6.204330205150654e-08
\end{verbatim}

We see that the problem arises at $x_{14}$ because in $x_{15}$, both $x_{14}$ and $x_{13}$ will be negative, which means that if you add them together, they will become more negative, which in turn means that $x_n$ will go towards $-\infty$.
Since we are dealing with small number in the printouts over, there must have been a rounding error that caused $x_{14}$ and $x_{13}$ to be negative, which in turn snowballed into a huge error when the program calculated $x_{100}$.

\newpage

\section*{Task 2}
\subsection*{a)}
\lstinputlisting[language=Python]{../oppgave_2.py}

The reason why it's better to use a floating point number, is because a float can represent 
much larger numbers than an integer can. \\
The results from the program is shown under: \\

\begin{quote}
    n =   5000, i =     4 ==> 26010428123750.0 \\
    n =   1000, i =   500 ==> 2.702882409454359e+299 \\
    n = 100000, i = 99940 ==> inf \\
\end{quote}

\subsection*{b)}
It is possible to get an overflow if the binomial coefficient is less than the maximum possible
floating point number that can be represented on my machine, because if one wants to calculate a binomial
coefficient where $i > n//2$ the coefficient will be smaller or equal than when $i = n//2$, and since the code essentially
calculates all of the coefficients from ${n \choose 1}$ to ${n \choose i}$, it can give an overflow during the calculation.
\subsection*{c)}
No, the method is not as effective for all values, because if we look at the equation for the binomial coefficient:
\begin{equation}
    {n \choose i} = \frac{n!}{i!(n-i)!}
\end{equation}
If we choose that $i = n-i$ we get:
\begin{equation}
    \begin{split}
        {n \choose n-i} &= \frac{n!}{(n-i)!(n-(n-i))!} \\
                        &= \frac{n!}{(n-i)!(n-n+i)!} \\
                        &= \frac{n!}{(n-i)!i!}
    \end{split}
\end{equation}
This means that ${n \choose i} = {n \choose n-i}$, which also means that if we want to calculate the binomial coefficient
of ${n \choose i}$ where $i > \frac{n}{2}$, then we can just calculate the binomial coefficient of ${n \choose n-i}$ and get the same answer.
This means that we can optimize our program by first choosing the lowest value of $i$ or $n-i$, and then calculate the binomial coefficient. Here is how the code would look like:

\lstinputlisting[language=Python]{../oppgave_2_v2.py}

\newpage

\section*{Task 3}
\subsection*{a)}
The program goes through a for loop where three random numbers between 0 and 1, and then compares $(x + y) \cdot z$ and $x \cdot z + y \cdot z$, which should yield
the same result since this is just the distributive law. It then checks if they are not equal, and if that statement is true, it increments a fault counter by 1.
The last thing the program does is to print out the error rate percentage, the last three random numbers chosen, and the error margin between the last two results.

\subsection*{b)}
\lstinputlisting[language=Python]{../oppgave_3.py}
When running the code over, the first number that was printed out was $41,577$. \\\\
One reason that the first number differs in the two comparisons could be that since the program uses random numbers, the "success rate" of the comparisons will vary slightly.
I think that the biggest reason why they differ is that since we are doing more multiplicative operations in the second program, the chance for the answers not coresponding
with each other is greater.




\end{document}